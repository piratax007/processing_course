
% Default to the notebook output style

    


% Inherit from the specified cell style.




    
\documentclass[11pt]{article}

    
    
    \usepackage[T1]{fontenc}
    % Nicer default font (+ math font) than Computer Modern for most use cases
    \usepackage{mathpazo}

    % Basic figure setup, for now with no caption control since it's done
    % automatically by Pandoc (which extracts ![](path) syntax from Markdown).
    \usepackage{graphicx}
    % We will generate all images so they have a width \maxwidth. This means
    % that they will get their normal width if they fit onto the page, but
    % are scaled down if they would overflow the margins.
    \makeatletter
    \def\maxwidth{\ifdim\Gin@nat@width>\linewidth\linewidth
    \else\Gin@nat@width\fi}
    \makeatother
    \let\Oldincludegraphics\includegraphics
    % Set max figure width to be 80% of text width, for now hardcoded.
    \renewcommand{\includegraphics}[1]{\Oldincludegraphics[width=.8\maxwidth]{#1}}
    % Ensure that by default, figures have no caption (until we provide a
    % proper Figure object with a Caption API and a way to capture that
    % in the conversion process - todo).
    \usepackage{caption}
    \DeclareCaptionLabelFormat{nolabel}{}
    \captionsetup{labelformat=nolabel}

    \usepackage{adjustbox} % Used to constrain images to a maximum size 
    \usepackage{xcolor} % Allow colors to be defined
    \usepackage{enumerate} % Needed for markdown enumerations to work
    \usepackage{geometry} % Used to adjust the document margins
    \usepackage{amsmath} % Equations
    \usepackage{amssymb} % Equations
    \usepackage{textcomp} % defines textquotesingle
    % Hack from http://tex.stackexchange.com/a/47451/13684:
    \AtBeginDocument{%
        \def\PYZsq{\textquotesingle}% Upright quotes in Pygmentized code
    }
    \usepackage{upquote} % Upright quotes for verbatim code
    \usepackage{eurosym} % defines \euro
    \usepackage[mathletters]{ucs} % Extended unicode (utf-8) support
    \usepackage[utf8x]{inputenc} % Allow utf-8 characters in the tex document
    \usepackage{fancyvrb} % verbatim replacement that allows latex
    \usepackage{grffile} % extends the file name processing of package graphics 
                         % to support a larger range 
    % The hyperref package gives us a pdf with properly built
    % internal navigation ('pdf bookmarks' for the table of contents,
    % internal cross-reference links, web links for URLs, etc.)
    \usepackage{hyperref}
    \usepackage{longtable} % longtable support required by pandoc >1.10
    \usepackage{booktabs}  % table support for pandoc > 1.12.2
    \usepackage[inline]{enumitem} % IRkernel/repr support (it uses the enumerate* environment)
    \usepackage[normalem]{ulem} % ulem is needed to support strikethroughs (\sout)
                                % normalem makes italics be italics, not underlines
    

    
    
    % Colors for the hyperref package
    \definecolor{urlcolor}{rgb}{0,.145,.698}
    \definecolor{linkcolor}{rgb}{.71,0.21,0.01}
    \definecolor{citecolor}{rgb}{.12,.54,.11}

    % ANSI colors
    \definecolor{ansi-black}{HTML}{3E424D}
    \definecolor{ansi-black-intense}{HTML}{282C36}
    \definecolor{ansi-red}{HTML}{E75C58}
    \definecolor{ansi-red-intense}{HTML}{B22B31}
    \definecolor{ansi-green}{HTML}{00A250}
    \definecolor{ansi-green-intense}{HTML}{007427}
    \definecolor{ansi-yellow}{HTML}{DDB62B}
    \definecolor{ansi-yellow-intense}{HTML}{B27D12}
    \definecolor{ansi-blue}{HTML}{208FFB}
    \definecolor{ansi-blue-intense}{HTML}{0065CA}
    \definecolor{ansi-magenta}{HTML}{D160C4}
    \definecolor{ansi-magenta-intense}{HTML}{A03196}
    \definecolor{ansi-cyan}{HTML}{60C6C8}
    \definecolor{ansi-cyan-intense}{HTML}{258F8F}
    \definecolor{ansi-white}{HTML}{C5C1B4}
    \definecolor{ansi-white-intense}{HTML}{A1A6B2}

    % commands and environments needed by pandoc snippets
    % extracted from the output of `pandoc -s`
    \providecommand{\tightlist}{%
      \setlength{\itemsep}{0pt}\setlength{\parskip}{0pt}}
    \DefineVerbatimEnvironment{Highlighting}{Verbatim}{commandchars=\\\{\}}
    % Add ',fontsize=\small' for more characters per line
    \newenvironment{Shaded}{}{}
    \newcommand{\KeywordTok}[1]{\textcolor[rgb]{0.00,0.44,0.13}{\textbf{{#1}}}}
    \newcommand{\DataTypeTok}[1]{\textcolor[rgb]{0.56,0.13,0.00}{{#1}}}
    \newcommand{\DecValTok}[1]{\textcolor[rgb]{0.25,0.63,0.44}{{#1}}}
    \newcommand{\BaseNTok}[1]{\textcolor[rgb]{0.25,0.63,0.44}{{#1}}}
    \newcommand{\FloatTok}[1]{\textcolor[rgb]{0.25,0.63,0.44}{{#1}}}
    \newcommand{\CharTok}[1]{\textcolor[rgb]{0.25,0.44,0.63}{{#1}}}
    \newcommand{\StringTok}[1]{\textcolor[rgb]{0.25,0.44,0.63}{{#1}}}
    \newcommand{\CommentTok}[1]{\textcolor[rgb]{0.38,0.63,0.69}{\textit{{#1}}}}
    \newcommand{\OtherTok}[1]{\textcolor[rgb]{0.00,0.44,0.13}{{#1}}}
    \newcommand{\AlertTok}[1]{\textcolor[rgb]{1.00,0.00,0.00}{\textbf{{#1}}}}
    \newcommand{\FunctionTok}[1]{\textcolor[rgb]{0.02,0.16,0.49}{{#1}}}
    \newcommand{\RegionMarkerTok}[1]{{#1}}
    \newcommand{\ErrorTok}[1]{\textcolor[rgb]{1.00,0.00,0.00}{\textbf{{#1}}}}
    \newcommand{\NormalTok}[1]{{#1}}
    
    % Additional commands for more recent versions of Pandoc
    \newcommand{\ConstantTok}[1]{\textcolor[rgb]{0.53,0.00,0.00}{{#1}}}
    \newcommand{\SpecialCharTok}[1]{\textcolor[rgb]{0.25,0.44,0.63}{{#1}}}
    \newcommand{\VerbatimStringTok}[1]{\textcolor[rgb]{0.25,0.44,0.63}{{#1}}}
    \newcommand{\SpecialStringTok}[1]{\textcolor[rgb]{0.73,0.40,0.53}{{#1}}}
    \newcommand{\ImportTok}[1]{{#1}}
    \newcommand{\DocumentationTok}[1]{\textcolor[rgb]{0.73,0.13,0.13}{\textit{{#1}}}}
    \newcommand{\AnnotationTok}[1]{\textcolor[rgb]{0.38,0.63,0.69}{\textbf{\textit{{#1}}}}}
    \newcommand{\CommentVarTok}[1]{\textcolor[rgb]{0.38,0.63,0.69}{\textbf{\textit{{#1}}}}}
    \newcommand{\VariableTok}[1]{\textcolor[rgb]{0.10,0.09,0.49}{{#1}}}
    \newcommand{\ControlFlowTok}[1]{\textcolor[rgb]{0.00,0.44,0.13}{\textbf{{#1}}}}
    \newcommand{\OperatorTok}[1]{\textcolor[rgb]{0.40,0.40,0.40}{{#1}}}
    \newcommand{\BuiltInTok}[1]{{#1}}
    \newcommand{\ExtensionTok}[1]{{#1}}
    \newcommand{\PreprocessorTok}[1]{\textcolor[rgb]{0.74,0.48,0.00}{{#1}}}
    \newcommand{\AttributeTok}[1]{\textcolor[rgb]{0.49,0.56,0.16}{{#1}}}
    \newcommand{\InformationTok}[1]{\textcolor[rgb]{0.38,0.63,0.69}{\textbf{\textit{{#1}}}}}
    \newcommand{\WarningTok}[1]{\textcolor[rgb]{0.38,0.63,0.69}{\textbf{\textit{{#1}}}}}
    
    
    % Define a nice break command that doesn't care if a line doesn't already
    % exist.
    \def\br{\hspace*{\fill} \\* }
    % Math Jax compatability definitions
    \def\gt{>}
    \def\lt{<}
    % Document parameters
    \title{Leccion\_7}
    
    
    

    % Pygments definitions
    
\makeatletter
\def\PY@reset{\let\PY@it=\relax \let\PY@bf=\relax%
    \let\PY@ul=\relax \let\PY@tc=\relax%
    \let\PY@bc=\relax \let\PY@ff=\relax}
\def\PY@tok#1{\csname PY@tok@#1\endcsname}
\def\PY@toks#1+{\ifx\relax#1\empty\else%
    \PY@tok{#1}\expandafter\PY@toks\fi}
\def\PY@do#1{\PY@bc{\PY@tc{\PY@ul{%
    \PY@it{\PY@bf{\PY@ff{#1}}}}}}}
\def\PY#1#2{\PY@reset\PY@toks#1+\relax+\PY@do{#2}}

\expandafter\def\csname PY@tok@w\endcsname{\def\PY@tc##1{\textcolor[rgb]{0.73,0.73,0.73}{##1}}}
\expandafter\def\csname PY@tok@c\endcsname{\let\PY@it=\textit\def\PY@tc##1{\textcolor[rgb]{0.25,0.50,0.50}{##1}}}
\expandafter\def\csname PY@tok@cp\endcsname{\def\PY@tc##1{\textcolor[rgb]{0.74,0.48,0.00}{##1}}}
\expandafter\def\csname PY@tok@k\endcsname{\let\PY@bf=\textbf\def\PY@tc##1{\textcolor[rgb]{0.00,0.50,0.00}{##1}}}
\expandafter\def\csname PY@tok@kp\endcsname{\def\PY@tc##1{\textcolor[rgb]{0.00,0.50,0.00}{##1}}}
\expandafter\def\csname PY@tok@kt\endcsname{\def\PY@tc##1{\textcolor[rgb]{0.69,0.00,0.25}{##1}}}
\expandafter\def\csname PY@tok@o\endcsname{\def\PY@tc##1{\textcolor[rgb]{0.40,0.40,0.40}{##1}}}
\expandafter\def\csname PY@tok@ow\endcsname{\let\PY@bf=\textbf\def\PY@tc##1{\textcolor[rgb]{0.67,0.13,1.00}{##1}}}
\expandafter\def\csname PY@tok@nb\endcsname{\def\PY@tc##1{\textcolor[rgb]{0.00,0.50,0.00}{##1}}}
\expandafter\def\csname PY@tok@nf\endcsname{\def\PY@tc##1{\textcolor[rgb]{0.00,0.00,1.00}{##1}}}
\expandafter\def\csname PY@tok@nc\endcsname{\let\PY@bf=\textbf\def\PY@tc##1{\textcolor[rgb]{0.00,0.00,1.00}{##1}}}
\expandafter\def\csname PY@tok@nn\endcsname{\let\PY@bf=\textbf\def\PY@tc##1{\textcolor[rgb]{0.00,0.00,1.00}{##1}}}
\expandafter\def\csname PY@tok@ne\endcsname{\let\PY@bf=\textbf\def\PY@tc##1{\textcolor[rgb]{0.82,0.25,0.23}{##1}}}
\expandafter\def\csname PY@tok@nv\endcsname{\def\PY@tc##1{\textcolor[rgb]{0.10,0.09,0.49}{##1}}}
\expandafter\def\csname PY@tok@no\endcsname{\def\PY@tc##1{\textcolor[rgb]{0.53,0.00,0.00}{##1}}}
\expandafter\def\csname PY@tok@nl\endcsname{\def\PY@tc##1{\textcolor[rgb]{0.63,0.63,0.00}{##1}}}
\expandafter\def\csname PY@tok@ni\endcsname{\let\PY@bf=\textbf\def\PY@tc##1{\textcolor[rgb]{0.60,0.60,0.60}{##1}}}
\expandafter\def\csname PY@tok@na\endcsname{\def\PY@tc##1{\textcolor[rgb]{0.49,0.56,0.16}{##1}}}
\expandafter\def\csname PY@tok@nt\endcsname{\let\PY@bf=\textbf\def\PY@tc##1{\textcolor[rgb]{0.00,0.50,0.00}{##1}}}
\expandafter\def\csname PY@tok@nd\endcsname{\def\PY@tc##1{\textcolor[rgb]{0.67,0.13,1.00}{##1}}}
\expandafter\def\csname PY@tok@s\endcsname{\def\PY@tc##1{\textcolor[rgb]{0.73,0.13,0.13}{##1}}}
\expandafter\def\csname PY@tok@sd\endcsname{\let\PY@it=\textit\def\PY@tc##1{\textcolor[rgb]{0.73,0.13,0.13}{##1}}}
\expandafter\def\csname PY@tok@si\endcsname{\let\PY@bf=\textbf\def\PY@tc##1{\textcolor[rgb]{0.73,0.40,0.53}{##1}}}
\expandafter\def\csname PY@tok@se\endcsname{\let\PY@bf=\textbf\def\PY@tc##1{\textcolor[rgb]{0.73,0.40,0.13}{##1}}}
\expandafter\def\csname PY@tok@sr\endcsname{\def\PY@tc##1{\textcolor[rgb]{0.73,0.40,0.53}{##1}}}
\expandafter\def\csname PY@tok@ss\endcsname{\def\PY@tc##1{\textcolor[rgb]{0.10,0.09,0.49}{##1}}}
\expandafter\def\csname PY@tok@sx\endcsname{\def\PY@tc##1{\textcolor[rgb]{0.00,0.50,0.00}{##1}}}
\expandafter\def\csname PY@tok@m\endcsname{\def\PY@tc##1{\textcolor[rgb]{0.40,0.40,0.40}{##1}}}
\expandafter\def\csname PY@tok@gh\endcsname{\let\PY@bf=\textbf\def\PY@tc##1{\textcolor[rgb]{0.00,0.00,0.50}{##1}}}
\expandafter\def\csname PY@tok@gu\endcsname{\let\PY@bf=\textbf\def\PY@tc##1{\textcolor[rgb]{0.50,0.00,0.50}{##1}}}
\expandafter\def\csname PY@tok@gd\endcsname{\def\PY@tc##1{\textcolor[rgb]{0.63,0.00,0.00}{##1}}}
\expandafter\def\csname PY@tok@gi\endcsname{\def\PY@tc##1{\textcolor[rgb]{0.00,0.63,0.00}{##1}}}
\expandafter\def\csname PY@tok@gr\endcsname{\def\PY@tc##1{\textcolor[rgb]{1.00,0.00,0.00}{##1}}}
\expandafter\def\csname PY@tok@ge\endcsname{\let\PY@it=\textit}
\expandafter\def\csname PY@tok@gs\endcsname{\let\PY@bf=\textbf}
\expandafter\def\csname PY@tok@gp\endcsname{\let\PY@bf=\textbf\def\PY@tc##1{\textcolor[rgb]{0.00,0.00,0.50}{##1}}}
\expandafter\def\csname PY@tok@go\endcsname{\def\PY@tc##1{\textcolor[rgb]{0.53,0.53,0.53}{##1}}}
\expandafter\def\csname PY@tok@gt\endcsname{\def\PY@tc##1{\textcolor[rgb]{0.00,0.27,0.87}{##1}}}
\expandafter\def\csname PY@tok@err\endcsname{\def\PY@bc##1{\setlength{\fboxsep}{0pt}\fcolorbox[rgb]{1.00,0.00,0.00}{1,1,1}{\strut ##1}}}
\expandafter\def\csname PY@tok@kc\endcsname{\let\PY@bf=\textbf\def\PY@tc##1{\textcolor[rgb]{0.00,0.50,0.00}{##1}}}
\expandafter\def\csname PY@tok@kd\endcsname{\let\PY@bf=\textbf\def\PY@tc##1{\textcolor[rgb]{0.00,0.50,0.00}{##1}}}
\expandafter\def\csname PY@tok@kn\endcsname{\let\PY@bf=\textbf\def\PY@tc##1{\textcolor[rgb]{0.00,0.50,0.00}{##1}}}
\expandafter\def\csname PY@tok@kr\endcsname{\let\PY@bf=\textbf\def\PY@tc##1{\textcolor[rgb]{0.00,0.50,0.00}{##1}}}
\expandafter\def\csname PY@tok@bp\endcsname{\def\PY@tc##1{\textcolor[rgb]{0.00,0.50,0.00}{##1}}}
\expandafter\def\csname PY@tok@fm\endcsname{\def\PY@tc##1{\textcolor[rgb]{0.00,0.00,1.00}{##1}}}
\expandafter\def\csname PY@tok@vc\endcsname{\def\PY@tc##1{\textcolor[rgb]{0.10,0.09,0.49}{##1}}}
\expandafter\def\csname PY@tok@vg\endcsname{\def\PY@tc##1{\textcolor[rgb]{0.10,0.09,0.49}{##1}}}
\expandafter\def\csname PY@tok@vi\endcsname{\def\PY@tc##1{\textcolor[rgb]{0.10,0.09,0.49}{##1}}}
\expandafter\def\csname PY@tok@vm\endcsname{\def\PY@tc##1{\textcolor[rgb]{0.10,0.09,0.49}{##1}}}
\expandafter\def\csname PY@tok@sa\endcsname{\def\PY@tc##1{\textcolor[rgb]{0.73,0.13,0.13}{##1}}}
\expandafter\def\csname PY@tok@sb\endcsname{\def\PY@tc##1{\textcolor[rgb]{0.73,0.13,0.13}{##1}}}
\expandafter\def\csname PY@tok@sc\endcsname{\def\PY@tc##1{\textcolor[rgb]{0.73,0.13,0.13}{##1}}}
\expandafter\def\csname PY@tok@dl\endcsname{\def\PY@tc##1{\textcolor[rgb]{0.73,0.13,0.13}{##1}}}
\expandafter\def\csname PY@tok@s2\endcsname{\def\PY@tc##1{\textcolor[rgb]{0.73,0.13,0.13}{##1}}}
\expandafter\def\csname PY@tok@sh\endcsname{\def\PY@tc##1{\textcolor[rgb]{0.73,0.13,0.13}{##1}}}
\expandafter\def\csname PY@tok@s1\endcsname{\def\PY@tc##1{\textcolor[rgb]{0.73,0.13,0.13}{##1}}}
\expandafter\def\csname PY@tok@mb\endcsname{\def\PY@tc##1{\textcolor[rgb]{0.40,0.40,0.40}{##1}}}
\expandafter\def\csname PY@tok@mf\endcsname{\def\PY@tc##1{\textcolor[rgb]{0.40,0.40,0.40}{##1}}}
\expandafter\def\csname PY@tok@mh\endcsname{\def\PY@tc##1{\textcolor[rgb]{0.40,0.40,0.40}{##1}}}
\expandafter\def\csname PY@tok@mi\endcsname{\def\PY@tc##1{\textcolor[rgb]{0.40,0.40,0.40}{##1}}}
\expandafter\def\csname PY@tok@il\endcsname{\def\PY@tc##1{\textcolor[rgb]{0.40,0.40,0.40}{##1}}}
\expandafter\def\csname PY@tok@mo\endcsname{\def\PY@tc##1{\textcolor[rgb]{0.40,0.40,0.40}{##1}}}
\expandafter\def\csname PY@tok@ch\endcsname{\let\PY@it=\textit\def\PY@tc##1{\textcolor[rgb]{0.25,0.50,0.50}{##1}}}
\expandafter\def\csname PY@tok@cm\endcsname{\let\PY@it=\textit\def\PY@tc##1{\textcolor[rgb]{0.25,0.50,0.50}{##1}}}
\expandafter\def\csname PY@tok@cpf\endcsname{\let\PY@it=\textit\def\PY@tc##1{\textcolor[rgb]{0.25,0.50,0.50}{##1}}}
\expandafter\def\csname PY@tok@c1\endcsname{\let\PY@it=\textit\def\PY@tc##1{\textcolor[rgb]{0.25,0.50,0.50}{##1}}}
\expandafter\def\csname PY@tok@cs\endcsname{\let\PY@it=\textit\def\PY@tc##1{\textcolor[rgb]{0.25,0.50,0.50}{##1}}}

\def\PYZbs{\char`\\}
\def\PYZus{\char`\_}
\def\PYZob{\char`\{}
\def\PYZcb{\char`\}}
\def\PYZca{\char`\^}
\def\PYZam{\char`\&}
\def\PYZlt{\char`\<}
\def\PYZgt{\char`\>}
\def\PYZsh{\char`\#}
\def\PYZpc{\char`\%}
\def\PYZdl{\char`\$}
\def\PYZhy{\char`\-}
\def\PYZsq{\char`\'}
\def\PYZdq{\char`\"}
\def\PYZti{\char`\~}
% for compatibility with earlier versions
\def\PYZat{@}
\def\PYZlb{[}
\def\PYZrb{]}
\makeatother


    % Exact colors from NB
    \definecolor{incolor}{rgb}{0.0, 0.0, 0.5}
    \definecolor{outcolor}{rgb}{0.545, 0.0, 0.0}



    
    % Prevent overflowing lines due to hard-to-break entities
    \sloppy 
    % Setup hyperref package
    \hypersetup{
      breaklinks=true,  % so long urls are correctly broken across lines
      colorlinks=true,
      urlcolor=urlcolor,
      linkcolor=linkcolor,
      citecolor=citecolor,
      }
    % Slightly bigger margins than the latex defaults
    
    \geometry{verbose,tmargin=1in,bmargin=1in,lmargin=1in,rmargin=1in}
    
    

    \begin{document}
    
    
    \maketitle
    
    

    
    \hypertarget{arreglos-vectores}{%
\section{Arreglos (vectores)}\label{arreglos-vectores}}

Tal como se vió en la
\href{https://nbviewer.jupyter.org/github/piratax007/processing_course/blob/1765d581fd750be2ae8c001e2bae4708b0d66f12/Leccion_2.ipynb\#Variables}{Lección
2 - Variables}, una varible es un espacio en memoria para almacenar un
dato (numéro, caracter, color, etc.) el uso de variables tiene tres
elementos 1. Definición: Reserva el espacio en memoria para un tipo de
dato específico e.g. \texttt{int\ variable;}. 2. Inicialización:
Determina el primer valor que albergará el espacio reservado, dicho
valor puede alterarse de forma dinámica. 3. Uso: El llamado a la
variable bien para leer el valor almacenado o para alterarlo.

\hypertarget{quuxe9-es-un-arreglo}{%
\subsection{¿Qué es un arreglo?}\label{quuxe9-es-un-arreglo}}

Los arreglos o \emph{vectores} son espacios en memoria destinados a
almacenar \texttt{n} elementos del mismo tipo, así por ejemplo se puede
tener un arreglo de números enteros almacenados en memoria con la misma
referencia pero diferente valor. Para utilizar arreglos se requieren
cuatro elementos 1. Definición: Determina el tipo de objeto que
contendrá cada espacio del arreglo. 2. (1.1) Creación: Establece la
longitud (\texttt{length}) del arreglo i.e.~cuántos espacios o elementos
almacenará. 3. Inicialización 4. Uso

Los elementos de un arreglo, debido a que están definidos bajo la misma
referencia o nombre, se identifican mediante un número \emph{índice}.
Así por ejemplo en el arreglo
\[M = \left[\matrix{-3 & 4 & 100 & 2 & -97}\right]\] los elementos
individuales son \(M_0 = -3\), \(M_1 = 4\), \ldots{}, \(M_4 = -97\). El
primer elemento de un arreglo siempre tendrá índice 0.

\hypertarget{cuuxe1ndo-utilizar-arreglos}{%
\subsection{¿Cuándo utilizar
arreglos?}\label{cuuxe1ndo-utilizar-arreglos}}

Piensa en esta situación: Quieres hacer un sketch en el que una elipse
cambie cada 10 segundos de color de forma aleatoria pero los colores
deben ser, azul, cyan, amarillo, naranja, rojo y negro ¿cómo haces esto?
evidentemente no puedes utilizar generación aleatoria de números para
establecer los componentes RGB de cada color porque será muy difíl
conseguir los mismos colores exactamente cada vez, lo mejor que puedes
hacer es definir un arreglo que contenga los seis colores que quieres
utilizar y seleccionar uno de estos cada vez de forma aleatoria. veamos
cómo

    \begin{Verbatim}[commandchars=\\\{\}]
{\color{incolor}In [{\color{incolor}1}]:} \PY{n}{color}\PY{o}{[}\PY{o}{]} \PY{n}{colors} \PY{o}{=} \PY{k}{new} \PY{n}{color}\PY{o}{[}\PY{l+m+mi}{6}\PY{o}{]}\PY{o}{;} \PY{c+c1}{// Crea un arreglo de 6 espacios del tipo color}
        \PY{k+kt}{int} \PY{n}{time} \PY{o}{=} \PY{n}{millis}\PY{o}{(}\PY{o}{)}\PY{o}{;}
        \PY{k+kt}{int} \PY{n}{selectColor} \PY{o}{=} \PY{k+kt}{int}\PY{o}{(}\PY{n}{random}\PY{o}{(}\PY{l+m+mi}{6}\PY{o}{)}\PY{o}{)}\PY{o}{;} \PY{c+c1}{// genera un número entero aleatorio entre 0 y 5}
        
        \PY{k+kt}{void} \PY{n+nf}{setup}\PY{o}{(}\PY{o}{)}\PY{o}{\PYZob{}}
            \PY{n}{colors}\PY{o}{[}\PY{l+m+mi}{0}\PY{o}{]} \PY{o}{=} \PY{n}{color}\PY{o}{(}\PY{err}{\PYZsh{}}\PY{l+m+mo}{0012}\PY{n}{DB}\PY{o}{)}\PY{o}{;} \PY{c+c1}{// Azul}
            \PY{n}{colors}\PY{o}{[}\PY{l+m+mi}{1}\PY{o}{]} \PY{o}{=} \PY{n}{color}\PY{o}{(}\PY{err}{\PYZsh{}}\PY{l+m+mo}{00}\PY{l+m+mi}{97}\PY{n}{DB}\PY{o}{)}\PY{o}{;} \PY{c+c1}{// Cyan}
            \PY{n}{colors}\PY{o}{[}\PY{l+m+mi}{2}\PY{o}{]} \PY{o}{=} \PY{n}{color}\PY{o}{(}\PY{err}{\PYZsh{}}\PY{n}{DBC900}\PY{o}{)}\PY{o}{;} \PY{c+c1}{// Amarillo}
            \PY{n}{colors}\PY{o}{[}\PY{l+m+mi}{3}\PY{o}{]} \PY{o}{=} \PY{n}{color}\PY{o}{(}\PY{err}{\PYZsh{}}\PY{n}{DB5700}\PY{o}{)}\PY{o}{;} \PY{c+c1}{// Naranja}
            \PY{n}{colors}\PY{o}{[}\PY{l+m+mi}{4}\PY{o}{]} \PY{o}{=} \PY{n}{color}\PY{o}{(}\PY{err}{\PYZsh{}}\PY{n}{DB0012}\PY{o}{)}\PY{o}{;} \PY{c+c1}{// Rojo}
            \PY{n}{colors}\PY{o}{[}\PY{l+m+mi}{5}\PY{o}{]} \PY{o}{=} \PY{n}{color}\PY{o}{(}\PY{err}{\PYZsh{}}\PY{l+m+mo}{000000}\PY{o}{)}\PY{o}{;} \PY{c+c1}{// Negro}
        \PY{o}{\PYZcb{}}
        
        \PY{k+kt}{void} \PY{n+nf}{draw}\PY{o}{(}\PY{o}{)}\PY{o}{\PYZob{}}
            \PY{n}{background}\PY{o}{(}\PY{l+m+mi}{75}\PY{o}{)}\PY{o}{;}
            \PY{k}{if} \PY{o}{(}\PY{n}{millis}\PY{o}{(}\PY{o}{)} \PY{o}{\PYZgt{}} \PY{n}{time} \PY{o}{+} \PY{l+m+mi}{10000}\PY{o}{)}\PY{o}{\PYZob{}}
                \PY{n}{selectColor} \PY{o}{=} \PY{k+kt}{int}\PY{o}{(}\PY{n}{random}\PY{o}{(}\PY{l+m+mi}{6}\PY{o}{)}\PY{o}{)}\PY{o}{;}
                \PY{n}{time} \PY{o}{=} \PY{n}{millis}\PY{o}{(}\PY{o}{)}\PY{o}{;}
            \PY{o}{\PYZcb{}}    
            \PY{n}{noStroke}\PY{o}{(}\PY{o}{)}\PY{o}{;}
            \PY{n}{fill}\PY{o}{(}\PY{n}{colors}\PY{o}{[}\PY{n}{selectColor}\PY{o}{]}\PY{o}{)}\PY{o}{;}
            \PY{n}{ellipse}\PY{o}{(}\PY{n}{width} \PY{o}{/} \PY{l+m+mi}{2}\PY{o}{,} \PY{n}{height} \PY{o}{/} \PY{l+m+mi}{2}\PY{o}{,} \PY{l+m+mi}{50}\PY{o}{,} \PY{l+m+mi}{50}\PY{o}{)}\PY{o}{;}
        \PY{o}{\PYZcb{}}
\end{Verbatim}


    
    \begin{verbatim}
<IPython.core.display.Javascript object>
    \end{verbatim}

    
    
    \begin{verbatim}
<IPython.core.display.HTML object>
    \end{verbatim}

    
    En el sketch anterior se ha utilizado un arreglo \texttt{colors} para
almacenar seis colores y elegir cada vez uno de éstos de forma aleatoria
mediante la generación de números enteros \emph{randomizados} entre 0 y
5.

Fíjate cómo en la primera línea se define el arreglo y se determina su
longitud en una sola línea mientras que en el bloque
\emph{\texttt{setup}} se esta determinando el color que se almacena en
cada posición del arreglo. Esto mismo puede hacerse en una sola
instrucción tal como la siguiente donde se crea e inicializa cada
entrada del arreglo.

\begin{verbatim}
color[] colors = {color(#0012DB), color(#0097DB), color(#DBC900), //
    color(#DB5700), color(#DB0012), color(#000000)
    };
\end{verbatim}

\hypertarget{arreglos-de-objetos}{%
\subsection{Arreglos de objetos}\label{arreglos-de-objetos}}

    Como en el ejemplo anterior, un arreglo puede ser útil para definir un
conjunto de elementos del mismo tipo y poder así ``llamarlos'' de una
forma más simple en el sketch, es decir, en lugar de crear seis
variables (una por cada color) un rreglo que los contiene todos hace más
fácil su selección principalmente porque es más fácil generar un número
(índice) aleatorio que el nombre de una variable. También pueden
utilizarse arreglos para optimizar el uso de objetos, en el ejemplo
\href{https://github.com/piratax007/processing_course/tree/master/Ejemplos/pelotas_locas}{pelotas
locas} puedes ver cómo se utiliza una clase para generar pelotas que
revotan por todo el lienzo colisionando entre ellas pero la pregunta es
¿cómo harías si quisieras agregar 10, 20, o 100?. Lee atentamente el
código del siguiente ejemplo que te será explicado luego.

\hypertarget{star-trek}{%
\subsubsection{Star Trek}\label{star-trek}}

¿Quieres viajar por el espacio a unos cuántos parsecs de velocidad?
vamos a hacerlo con processing

    \begin{Verbatim}[commandchars=\\\{\}]
{\color{incolor}In [{\color{incolor}2}]:} \PY{n}{star}\PY{o}{[}\PY{o}{]} \PY{n}{estrellas} \PY{o}{=} \PY{k}{new} \PY{n}{star}\PY{o}{[}\PY{l+m+mi}{400}\PY{o}{]}\PY{o}{;}
        
        \PY{k+kt}{void} \PY{n+nf}{setup}\PY{o}{(}\PY{o}{)}\PY{o}{\PYZob{}}
            \PY{n}{size}\PY{o}{(}\PY{l+m+mi}{300}\PY{o}{,} \PY{l+m+mi}{300}\PY{o}{)}\PY{o}{;}
            \PY{k}{for} \PY{o}{(}\PY{k+kt}{int} \PY{n}{i} \PY{o}{=} \PY{l+m+mi}{0}\PY{o}{;} \PY{n}{i} \PY{o}{\PYZlt{}} \PY{n}{estrellas}\PY{o}{.}\PY{n+na}{length}\PY{o}{;} \PY{n}{i}\PY{o}{+}\PY{o}{+}\PY{o}{)}\PY{o}{\PYZob{}}
                \PY{n}{estrellas}\PY{o}{[}\PY{n}{i}\PY{o}{]} \PY{o}{=} \PY{k}{new} \PY{n}{star}\PY{o}{(}\PY{o}{)}\PY{o}{;}
            \PY{o}{\PYZcb{}}
        \PY{o}{\PYZcb{}}
        
        \PY{k+kt}{void} \PY{n+nf}{draw}\PY{o}{(}\PY{o}{)}\PY{o}{\PYZob{}}
            \PY{n}{background}\PY{o}{(}\PY{l+m+mi}{0}\PY{o}{)}\PY{o}{;}
            \PY{k}{for} \PY{o}{(}\PY{k+kt}{int} \PY{n}{i} \PY{o}{=} \PY{l+m+mi}{0}\PY{o}{;} \PY{n}{i} \PY{o}{\PYZlt{}} \PY{n}{estrellas}\PY{o}{.}\PY{n+na}{length}\PY{o}{;} \PY{n}{i}\PY{o}{+}\PY{o}{+}\PY{o}{)}\PY{o}{\PYZob{}}
                \PY{n}{estrellas}\PY{o}{[}\PY{n}{i}\PY{o}{]}\PY{o}{.}\PY{n+na}{dibujar}\PY{o}{(}\PY{o}{)}\PY{o}{;}
                \PY{n}{estrellas}\PY{o}{[}\PY{n}{i}\PY{o}{]}\PY{o}{.}\PY{n+na}{mover}\PY{o}{(}\PY{o}{)}\PY{o}{;}
            \PY{o}{\PYZcb{}}
        \PY{o}{\PYZcb{}}
        
        \PY{k+kd}{class} \PY{n+nc}{star}\PY{o}{\PYZob{}}
            \PY{k+kt}{float} \PY{n}{x}\PY{o}{,} \PY{n}{y}\PY{o}{,} \PY{n}{z}\PY{o}{;}
            
            \PY{n}{star}\PY{o}{(}\PY{o}{)}\PY{o}{\PYZob{}}
                \PY{n}{x} \PY{o}{=} \PY{n}{random}\PY{o}{(}\PY{o}{\PYZhy{}}\PY{n}{width}\PY{o}{,} \PY{n}{width}\PY{o}{)}\PY{o}{;}
                \PY{n}{y} \PY{o}{=} \PY{n}{random}\PY{o}{(}\PY{o}{\PYZhy{}}\PY{n}{height}\PY{o}{,} \PY{n}{height}\PY{o}{)}\PY{o}{;}
                \PY{n}{z} \PY{o}{=} \PY{n}{random}\PY{o}{(}\PY{n}{width}\PY{o}{)}\PY{o}{;}
            \PY{o}{\PYZcb{}}
            
            \PY{k+kt}{void} \PY{n+nf}{dibujar}\PY{o}{(}\PY{o}{)}\PY{o}{\PYZob{}}
                \PY{k+kt}{float} \PY{n}{sx} \PY{o}{=} \PY{n}{map}\PY{o}{(}\PY{n}{x} \PY{o}{/} \PY{n}{z}\PY{o}{,} \PY{l+m+mi}{0}\PY{o}{,} \PY{l+m+mi}{1}\PY{o}{,} \PY{n}{width} \PY{o}{/} \PY{l+m+mi}{2}\PY{o}{,} \PY{n}{width}\PY{o}{)}\PY{o}{;}
                \PY{k+kt}{float} \PY{n}{sy} \PY{o}{=} \PY{n}{map}\PY{o}{(}\PY{n}{y} \PY{o}{/} \PY{n}{z}\PY{o}{,} \PY{l+m+mi}{0}\PY{o}{,} \PY{l+m+mi}{1}\PY{o}{,} \PY{n}{height} \PY{o}{/} \PY{l+m+mi}{2}\PY{o}{,} \PY{n}{height}\PY{o}{)}\PY{o}{;}
                \PY{k+kt}{float} \PY{n}{r} \PY{o}{=} \PY{n}{map}\PY{o}{(}\PY{n}{z}\PY{o}{,} \PY{l+m+mi}{0}\PY{o}{,} \PY{n}{width}\PY{o}{,} \PY{l+m+mi}{8}\PY{o}{,} \PY{l+m+mi}{0}\PY{o}{)}\PY{o}{;}
                
                \PY{n}{fill}\PY{o}{(}\PY{l+m+mi}{255}\PY{o}{)}\PY{o}{;}
                \PY{n}{noStroke}\PY{o}{(}\PY{o}{)}\PY{o}{;}
                \PY{n}{ellipse}\PY{o}{(}\PY{n}{sx}\PY{o}{,} \PY{n}{sy}\PY{o}{,} \PY{n}{r}\PY{o}{,} \PY{n}{r}\PY{o}{)}\PY{o}{;}
            \PY{o}{\PYZcb{}}
            
            \PY{k+kt}{void} \PY{n+nf}{mover}\PY{o}{(}\PY{o}{)}\PY{o}{\PYZob{}}
                \PY{n}{z} \PY{o}{\PYZhy{}}\PY{o}{=} \PY{l+m+mi}{10}\PY{o}{;}
                \PY{k}{if} \PY{o}{(}\PY{n}{z} \PY{o}{\PYZlt{}} \PY{l+m+mi}{1}\PY{o}{)}\PY{o}{\PYZob{}}
                    \PY{n}{z} \PY{o}{=} \PY{n}{random}\PY{o}{(}\PY{n}{width}\PY{o}{)}\PY{o}{;}
                    \PY{n}{x} \PY{o}{=} \PY{n}{random}\PY{o}{(}\PY{o}{\PYZhy{}}\PY{n}{width}\PY{o}{,} \PY{n}{width}\PY{o}{)}\PY{o}{;}
                    \PY{n}{y} \PY{o}{=} \PY{n}{random}\PY{o}{(}\PY{o}{\PYZhy{}}\PY{n}{height}\PY{o}{,} \PY{n}{height}\PY{o}{)}\PY{o}{;}
                \PY{o}{\PYZcb{}}
            \PY{o}{\PYZcb{}}
        \PY{o}{\PYZcb{}}
\end{Verbatim}


    
    \begin{verbatim}
<IPython.core.display.Javascript object>
    \end{verbatim}

    
    
    \begin{verbatim}
<IPython.core.display.HTML object>
    \end{verbatim}

    
    \hypertarget{descifrando-star-trek}{%
\subsubsection{Descifrando Star Trek}\label{descifrando-star-trek}}

No explicaremos el código de la clase \texttt{star} ahora que sabes
desarrollar clases puedes estudiarlo por ti mismo pero presta especial
atención a la función \texttt{map} que convierte un valor dado en un
rango específico a otro rango ascendente o descendente, lee en la
referencia acerca de esta función.

En el sketch, la primera línea

\begin{verbatim}
    star[] estrellas = new star[400];
\end{verbatim}

define un arreglo con cuatrocientas entradas de objetos del tipo
\texttt{star}. Cada una de estas estrellas se crea entre las líneas 5 y
7

\begin{verbatim}
    for (int i = 0; i < estrellas.length; i++){
        estrellas[i] = new star();
    }
\end{verbatim}

el bucle \texttt{for} recorre el arreglo desde la entrada 0
(\texttt{i\ =\ 0}) hasta la última entrada identificada con la
instrucción \texttt{lenght} que devuelve la longitud del arreglo, de una
en una creando un nuevo objeto de la clase \texttt{star} en cada
entrada. El mismo bucle entre las líneas 12 y 15 ejecuta las funciones
\texttt{dibujar} y \texttt{mover} de cada objeto \texttt{star}.

\begin{verbatim}
    for (int i = 0; i < estrellas.length; i++){
        estrellas[i].dibujar();
        estrellas[i].mover();
    }
\end{verbatim}

Evidentemente, con apenas algunas lineas de código puedes controlar
todos los objetos de la misma clase que necesites, si quieres más o
menos estrellas solo necesitas modificar la longitud del arreglo
\texttt{estrellas} en la primera línea del sketch.

\hypertarget{ejercicio-1}{%
\paragraph{Ejercicio 1:}\label{ejercicio-1}}

Descarga el ejemplo
\href{https://github.com/piratax007/processing_course/tree/master/Ejemplos/pelotas_locas}{pelotas
locas} y modifica el sketch de tal forma que mediante un arreglo de
objetos puedas controlar 10, 20 o el número de pelotas locas que quieras
modificando únicamente el tamaño del arreglo.

\hypertarget{arreglos-de-longitud-flexible}{%
\subsection{Arreglos de Longitud
Flexible}\label{arreglos-de-longitud-flexible}}

Hasta ahora has aprendido a definir arreglo de longitud fija. En el
ejemplo \texttt{Start\ Trek} tienes 400 estrellas o más o menos de
acuerdo a como definas la longitud del array, sin embargo, en ocasiones
puede que definir la longitud del array en un inicio sea algo que no se
pueda determinar o a lo menos difícil debido a que no puedes tener
siempre la certeza de cuántos objetos vas a utilizar y entonces es
cuando necesitas poder controlar la longitud de tus array en
\emph{tiempo de ejecución}. Puedes aplicar diversas alternativas para
conseguir esto, la menos adecuada es definir un array con una longitud
muy grande para así tener suficientes entradas de las que disponer e ir
creando o eliminando cada uno a medida que necesitas más o menos
elementos en el array, esta opción es muy inadecuada ya que significa
asignar un espacio en memoria que tal vez no utilices del todo por lo
tanto estarás consumiendo recursos innecesariamente. Otra opción es
utilizar el método \texttt{append} para añadir nuevas entradas a tu
array que después no podrás eliminar. El método \texttt{append} puede
hacer lenta la ejecución de tu sketch debido a que al agregar una
entrada no se hace sobre el array original sino que se crea una copia
del array con las entradas ya definidas más la nueva entrada creada,
así, si agregas 10 entradas nuevas mediante el método \texttt{append}
tendras 10 copias en memoria ocupando nuevamente recursos de forma
innecesaria.

\hypertarget{arraylist}{%
\subsubsection{ArrayList}\label{arraylist}}

Un \texttt{ArrayList} es básicamente un arreglo de longitud flexible al
cual se pueden agregar nuevas entradas o eliminar entradas sin generar
copias del mismo y de manera muy fácil.

Para definir un \texttt{ArrayList} puede utilizarse la sintaxis

\begin{verbatim}
ArrayList<tipo> nombre = new ArrayList<tipo>(tamaño inicial);
\end{verbatim}

Los principales métodos asociados a un \texttt{ArrayList} son
\texttt{size} equivalente a \texttt{legth} en un array de longitud fija,
\texttt{get} para referirse a una entreda particular, \texttt{add} para
agregar una nueva entrada al \texttt{ArrayList} y \texttt{remove} para
eliminar una entrada del \texttt{ArrayList}.

En el siguiente ejemplo se ha modificado el sketch el ejemplo
\href{https://github.com/piratax007/processing_course/tree/master/Ejemplos/pelotas_locas}{pelotas
locas} para que se agregue un nuevo objeto cada vez que se da click con
el mouse y se eliminen los objetos que colicionan

    \begin{Verbatim}[commandchars=\\\{\}]
{\color{incolor}In [{\color{incolor}3}]:} \PY{n}{ArrayList}\PY{o}{\PYZlt{}}\PY{n}{pelotaLoca}\PY{o}{\PYZgt{}} \PY{n}{bubbles} \PY{o}{=} \PY{k}{new} \PY{n}{ArrayList}\PY{o}{\PYZlt{}}\PY{n}{pelotaLoca}\PY{o}{\PYZgt{}}\PY{o}{(}\PY{o}{)}\PY{o}{;}
        
        \PY{k+kt}{void} \PY{n+nf}{setup}\PY{o}{(}\PY{o}{)}\PY{o}{\PYZob{}}
          \PY{n}{size}\PY{o}{(}\PY{l+m+mi}{400}\PY{o}{,} \PY{l+m+mi}{400}\PY{o}{)}\PY{o}{;}
        \PY{o}{\PYZcb{}}
        
        \PY{k+kt}{void} \PY{n+nf}{draw}\PY{o}{(}\PY{o}{)}\PY{o}{\PYZob{}}
          \PY{n}{background}\PY{o}{(}\PY{l+m+mi}{75}\PY{o}{)}\PY{o}{;}
          \PY{k}{for} \PY{o}{(}\PY{k+kt}{int} \PY{n}{i} \PY{o}{=} \PY{l+m+mi}{0}\PY{o}{;} \PY{n}{i} \PY{o}{\PYZlt{}} \PY{n}{bubbles}\PY{o}{.}\PY{n+na}{size}\PY{o}{(}\PY{o}{)} \PY{o}{\PYZhy{}} \PY{l+m+mi}{2}\PY{o}{;} \PY{n}{i}\PY{o}{+}\PY{o}{+}\PY{o}{)}\PY{o}{\PYZob{}}
            \PY{n}{bubbles}\PY{o}{.}\PY{n+na}{get}\PY{o}{(}\PY{n}{i}\PY{o}{)}\PY{o}{.}\PY{n+na}{dibujar}\PY{o}{(}\PY{o}{)}\PY{o}{;}
            \PY{n}{bubbles}\PY{o}{.}\PY{n+na}{get}\PY{o}{(}\PY{n}{i}\PY{o}{)}\PY{o}{.}\PY{n+na}{mover}\PY{o}{(}\PY{o}{)}\PY{o}{;}
            \PY{k}{for} \PY{o}{(}\PY{k+kt}{int} \PY{n}{j} \PY{o}{=} \PY{l+m+mi}{0}\PY{o}{;} \PY{n}{j} \PY{o}{\PYZlt{}} \PY{n}{bubbles}\PY{o}{.}\PY{n+na}{size}\PY{o}{(}\PY{o}{)} \PY{o}{\PYZhy{}} \PY{l+m+mi}{2}\PY{o}{;} \PY{n}{j}\PY{o}{+}\PY{o}{+}\PY{o}{)}\PY{o}{\PYZob{}}
              \PY{k}{if} \PY{o}{(}\PY{n}{i} \PY{o}{!}\PY{o}{=} \PY{n}{j} \PY{o}{\PYZam{}}\PY{o}{\PYZam{}} \PY{n}{bubbles}\PY{o}{.}\PY{n+na}{get}\PY{o}{(}\PY{n}{j}\PY{o}{)}\PY{o}{.}\PY{n+na}{colisionar}\PY{o}{(}\PY{n}{bubbles}\PY{o}{.}\PY{n+na}{get}\PY{o}{(}\PY{n}{i}\PY{o}{)}\PY{o}{,} \PY{n}{bubbles}\PY{o}{.}\PY{n+na}{get}\PY{o}{(}\PY{n}{j}\PY{o}{)}\PY{o}{)}\PY{o}{)}\PY{o}{\PYZob{}}
                \PY{n}{bubbles}\PY{o}{.}\PY{n+na}{remove}\PY{o}{(}\PY{n}{bubbles}\PY{o}{.}\PY{n+na}{get}\PY{o}{(}\PY{n}{i}\PY{o}{)}\PY{o}{)}\PY{o}{;}
                \PY{n}{bubbles}\PY{o}{.}\PY{n+na}{remove}\PY{o}{(}\PY{n}{bubbles}\PY{o}{.}\PY{n+na}{get}\PY{o}{(}\PY{n}{j}\PY{o}{)}\PY{o}{)}\PY{o}{;}
              \PY{o}{\PYZcb{}}
            \PY{o}{\PYZcb{}}
          \PY{o}{\PYZcb{}}
        \PY{o}{\PYZcb{}}
        
        \PY{k+kt}{void} \PY{n+nf}{mousePressed}\PY{o}{(}\PY{o}{)}\PY{o}{\PYZob{}}
          \PY{n}{bubbles}\PY{o}{.}\PY{n+na}{add}\PY{o}{(}\PY{k}{new} \PY{n}{pelotaLoca}\PY{o}{(}\PY{o}{)}\PY{o}{)}\PY{o}{;}
        \PY{o}{\PYZcb{}}
        
        \PY{k+kd}{class} \PY{n+nc}{pelotaLoca}\PY{o}{\PYZob{}}
          \PY{k+kt}{float} \PY{n}{h}\PY{o}{,} \PY{n}{k}\PY{o}{,} \PY{n}{r}\PY{o}{;}
          \PY{n}{color} \PY{n}{relleno}\PY{o}{;}
          \PY{k+kt}{float} \PY{n}{vX}\PY{o}{,} \PY{n}{vY}\PY{o}{;}
          
          \PY{n}{pelotaLoca}\PY{o}{(}\PY{o}{)}\PY{o}{\PYZob{}}
            \PY{n}{r} \PY{o}{=} \PY{n}{random}\PY{o}{(}\PY{l+m+mi}{10}\PY{o}{,} \PY{l+m+mi}{15}\PY{o}{)}\PY{o}{;}
            \PY{n}{h} \PY{o}{=} \PY{n}{random}\PY{o}{(}\PY{n}{r}\PY{o}{,} \PY{n}{width} \PY{o}{\PYZhy{}} \PY{n}{r}\PY{o}{)}\PY{o}{;}
            \PY{n}{k} \PY{o}{=} \PY{n}{random}\PY{o}{(}\PY{n}{r}\PY{o}{,} \PY{n}{height} \PY{o}{\PYZhy{}} \PY{n}{r}\PY{o}{)}\PY{o}{;}
            \PY{n}{relleno} \PY{o}{=} \PY{n}{color}\PY{o}{(}\PY{n}{random}\PY{o}{(}\PY{l+m+mi}{255}\PY{o}{)}\PY{o}{,} \PY{n}{random}\PY{o}{(}\PY{l+m+mi}{255}\PY{o}{)}\PY{o}{,} \PY{n}{random}\PY{o}{(}\PY{l+m+mi}{255}\PY{o}{)}\PY{o}{)}\PY{o}{;}
            \PY{n}{vX} \PY{o}{=} \PY{n}{random}\PY{o}{(}\PY{l+m+mi}{3}\PY{o}{,} \PY{l+m+mi}{5}\PY{o}{)}\PY{o}{;}
            \PY{n}{vY} \PY{o}{=} \PY{n}{random}\PY{o}{(}\PY{l+m+mi}{3}\PY{o}{,} \PY{l+m+mi}{5}\PY{o}{)}\PY{o}{;}
          \PY{o}{\PYZcb{}}
          
          \PY{n}{pelotaLoca}\PY{o}{(}\PY{k+kt}{float} \PY{n}{Radio}\PY{o}{,} \PY{n}{color} \PY{n}{Color}\PY{o}{)}\PY{o}{\PYZob{}}
            \PY{n}{r} \PY{o}{=} \PY{n}{Radio}\PY{o}{;}
            \PY{n}{h} \PY{o}{=} \PY{n}{random}\PY{o}{(}\PY{n}{r}\PY{o}{,} \PY{n}{width} \PY{o}{\PYZhy{}} \PY{n}{r}\PY{o}{)}\PY{o}{;}
            \PY{n}{k} \PY{o}{=} \PY{n}{random}\PY{o}{(}\PY{n}{r}\PY{o}{,} \PY{n}{height} \PY{o}{\PYZhy{}} \PY{n}{r}\PY{o}{)}\PY{o}{;}
            \PY{n}{relleno} \PY{o}{=} \PY{n}{Color}\PY{o}{;}
            \PY{n}{vX} \PY{o}{=} \PY{n}{random}\PY{o}{(}\PY{l+m+mi}{3}\PY{o}{,} \PY{l+m+mi}{5}\PY{o}{)}\PY{o}{;}
            \PY{n}{vY} \PY{o}{=} \PY{n}{random}\PY{o}{(}\PY{l+m+mi}{3}\PY{o}{,} \PY{l+m+mi}{5}\PY{o}{)}\PY{o}{;}
          \PY{o}{\PYZcb{}}
          
          \PY{k+kt}{void} \PY{n+nf}{dibujar}\PY{o}{(}\PY{o}{)}\PY{o}{\PYZob{}}
            \PY{n}{fill}\PY{o}{(}\PY{n}{relleno}\PY{o}{)}\PY{o}{;}
            \PY{n}{ellipse}\PY{o}{(}\PY{n}{h}\PY{o}{,} \PY{n}{k}\PY{o}{,} \PY{l+m+mi}{2} \PY{o}{*} \PY{n}{r}\PY{o}{,} \PY{l+m+mi}{2} \PY{o}{*} \PY{n}{r}\PY{o}{)}\PY{o}{;}
          \PY{o}{\PYZcb{}}
          
          \PY{k+kt}{void} \PY{n+nf}{mover}\PY{o}{(}\PY{o}{)}\PY{o}{\PYZob{}}
            \PY{n}{h} \PY{o}{+}\PY{o}{=} \PY{n}{vX}\PY{o}{;}
            \PY{n}{k} \PY{o}{+}\PY{o}{=} \PY{n}{vY}\PY{o}{;}
            
            \PY{k}{if} \PY{o}{(}\PY{n}{h} \PY{o}{\PYZlt{}} \PY{n}{r} \PY{o}{|}\PY{o}{|} \PY{n}{h} \PY{o}{\PYZgt{}} \PY{n}{width} \PY{o}{\PYZhy{}} \PY{n}{r}\PY{o}{)}\PY{o}{\PYZob{}}
              \PY{n}{vX} \PY{o}{*}\PY{o}{=} \PY{o}{\PYZhy{}}\PY{l+m+mi}{1}\PY{o}{;}
            \PY{o}{\PYZcb{}}
            
            \PY{k}{if} \PY{o}{(}\PY{n}{k} \PY{o}{\PYZlt{}} \PY{n}{r} \PY{o}{|}\PY{o}{|} \PY{n}{k} \PY{o}{\PYZgt{}} \PY{n}{height} \PY{o}{\PYZhy{}} \PY{n}{r}\PY{o}{)}\PY{o}{\PYZob{}}
              \PY{n}{vY} \PY{o}{*}\PY{o}{=} \PY{o}{\PYZhy{}}\PY{l+m+mi}{1}\PY{o}{;}
            \PY{o}{\PYZcb{}}
          \PY{o}{\PYZcb{}}
          
          \PY{k+kt}{boolean} \PY{n+nf}{colisionar}\PY{o}{(}\PY{n}{pelotaLoca} \PY{n}{P1}\PY{o}{,} \PY{n}{pelotaLoca} \PY{n}{P2}\PY{o}{)}\PY{o}{\PYZob{}}
            \PY{k+kt}{float} \PY{n}{vXP1} \PY{o}{=} \PY{n}{P1}\PY{o}{.}\PY{n+na}{vX}\PY{o}{;}
            \PY{k+kt}{float} \PY{n}{vYP1} \PY{o}{=} \PY{n}{P1}\PY{o}{.}\PY{n+na}{vY}\PY{o}{;}
            \PY{k+kt}{float} \PY{n}{vXP2} \PY{o}{=} \PY{n}{P2}\PY{o}{.}\PY{n+na}{vX}\PY{o}{;}
            \PY{k+kt}{float} \PY{n}{vYP2} \PY{o}{=} \PY{n}{P2}\PY{o}{.}\PY{n+na}{vY}\PY{o}{;}
            
            \PY{k}{if} \PY{o}{(}\PY{n}{dist}\PY{o}{(}\PY{n}{P1}\PY{o}{.}\PY{n+na}{h}\PY{o}{,} \PY{n}{P1}\PY{o}{.}\PY{n+na}{k}\PY{o}{,} \PY{n}{P2}\PY{o}{.}\PY{n+na}{h}\PY{o}{,} \PY{n}{P2}\PY{o}{.}\PY{n+na}{k}\PY{o}{)} \PY{o}{\PYZlt{}} \PY{n}{P1}\PY{o}{.}\PY{n+na}{r} \PY{o}{+} \PY{n}{P2}\PY{o}{.}\PY{n+na}{r}\PY{o}{)}\PY{o}{\PYZob{}}
              \PY{n}{P1}\PY{o}{.}\PY{n+na}{vX} \PY{o}{=} \PY{n}{vXP2}\PY{o}{;}
              \PY{n}{P1}\PY{o}{.}\PY{n+na}{vY} \PY{o}{=} \PY{n}{vYP2}\PY{o}{;}
              \PY{n}{P2}\PY{o}{.}\PY{n+na}{vX} \PY{o}{=} \PY{n}{vXP1}\PY{o}{;}
              \PY{n}{P2}\PY{o}{.}\PY{n+na}{vY} \PY{o}{=} \PY{n}{vYP1}\PY{o}{;}
              \PY{k}{return} \PY{k+kc}{true}\PY{o}{;}
            \PY{o}{\PYZcb{}}\PY{k}{else}\PY{o}{\PYZob{}}
              \PY{k}{return} \PY{k+kc}{false}\PY{o}{;}
            \PY{o}{\PYZcb{}}
          \PY{o}{\PYZcb{}}
        \PY{o}{\PYZcb{}}
\end{Verbatim}


    
    \begin{verbatim}
<IPython.core.display.Javascript object>
    \end{verbatim}

    
    
    \begin{verbatim}
<IPython.core.display.HTML object>
    \end{verbatim}

    
    \hypertarget{ejercicio-2}{%
\paragraph{Ejercicio 2}\label{ejercicio-2}}

Modifica el sketch anterior de tal forma que cuando dos pelotas locas
colisionen aparezca una nueva y si quieres retarte modifica otra versión
del mismo sketch para que cuando dos pelotas locas colisionen se
eliminen y aparezca una nueva con radio igual a la suma de los radios de
las que colicionaron, la nueva pelota loca debería aparecer en la
posición de la colisión.


    % Add a bibliography block to the postdoc
    
    
    
    \end{document}
